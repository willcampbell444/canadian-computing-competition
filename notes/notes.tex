\documentclass[12pt]{article}

\title{Canadian Computing Contest Notes}
\author{Will Campbell}

\usepackage{amsmath}
\usepackage[margin=1in]{geometry}
\usepackage{indentfirst}

\usepackage{listings}
\usepackage{color}

\definecolor{dkgreen}{rgb}{0,0.6,0}
\definecolor{gray}{rgb}{0.5,0.5,0.5}
\definecolor{mauve}{rgb}{0.58,0,0.82}

\lstset{frame=tb,
  language=Python,
  aboveskip=3mm,
  belowskip=3mm,
  showstringspaces=false,
  columns=flexible,
  basicstyle={\small\ttfamily},
  numbers=none,
  numberstyle=\tiny\color{gray},
  keywordstyle=\color{blue},
  commentstyle=\color{dkgreen},
  stringstyle=\color{mauve},
  breaklines=true,
  breakatwhitespace=true,
  tabsize=3
}

\begin{document}
\section{Miscellaneous}
\begin{lstlisting}
"{0:.2f}".format(number)
# returns number rounded to two decimal places
\end{lstlisting}

\section{Basic Containers}
\subsection{Lists}
\begin{lstlisting}
my_list = []

my_list[index]
# returns the item at the given index

my_list[-1]
# returns the last item,
my_list[-2]
# returns the second last item

my_list[start:end:step]
# returns a list of all items beginning at the given starting point up to but not including the end point, counting by intervals of step
# if start is left blank, it starts from the very beginning
# if end is left blank, it goes until the very end
# if step is left blank it goes by steps of +1

my_list.append(value)
# adds x to the end of the list

my_list.extend(other_list)
# adds all the items in other_list to the end of my_list

my_list.insert(index, value)
# adds an item into the list before the item at the given index

my_list.remove(x)
# removes the first item with value x
# if it does not exist an error will occour

my_list.pop(index)
# removes and returns the item at the given index
# if no index is given, the last item will be popped

my_list.clear()
# removes all items in the list

my_list.index(value[, start[, end]])
# returns the index of the first item with the given value
# error if none is found

my_list.count(value)
# returns the number of times the given value appears in the list

my_list.sort(key=None, reverse=False)
# sorts the list

my_list.sort(key=lambda x: x[2])
# sorts the list by the third element of each item in the list

my_list.reverse()
# reverses the list

my_list.copy()
# or
my_list[:]
# returns a shallow copy
\end{lstlisting}

\subsection{Stacks}
\begin{lstlisting}
stack = []

# push:
stack.append(value)

# pop:
stack.pop()
\end{lstlisting}

\subsection{Queues}
\begin{lstlisting}
from collections import deque

queue = deque()

# enqueue:
queue.append(value)

# dequeue:
queue.popleft()
\end{lstlisting}

\subsection{Priority Queues}
\begin{lstlisting}
import heapq

class PriorityQueueSimple(list):
	def insert(self, item, priority):
		heapq.heappush(self, (priority, item))

	def popmin(self):
		return heapq.heappop(self) # tuple of (priority, item)
\end{lstlisting}

\subsection{Dictionaries}
\begin{lstlisting}
my_dict = {}

del my_dict[key]
# removes the element with the given key

key in my_dict
# True if key is a key in the dictionary

my_dict.items()
# returns a view of (key, value) pairs

my_dict.keys()
# returns a view of the keys in the dictionary

my_dict.pop(key)
# removes and returns the item with the given key

my_dict.setdefault(key[, default])
# if key is in the dictionary, return its value
# if not, insert key with a value of default and return default
# default defaults to None

my_dict.update(other_dict)
# update my_dict to hold all the key, value pairs in other_dict

my_dict.values()
# returns a view of the values in the dict
\end{lstlisting}

\subsection{Sets}
\begin{lstlisting}
my_set = set()
my_set = {"a", "bb", "etc"}

value in my_set
# checks if value is in the set

my_set.isdisjoint(other_set)
# True if the sets do not share any elements

my_set.issubset(other_set)
# True if every element in the my_set is in other_set

my_set | other_set
# returns a set of everything in either set

my_set & other_set
# returns a set of every item that appears in both sets

my_set - other_set
# returns a set of elements that are in my_set but not in other_set

my_set ^ other_set
# returns a set of any item that only appears in one of the sets, not both

my_set.copy()
# returns a shallow copy of my_set

my_set.add(value)
# adds value to the set

my_set.remove(value)
# removes value from the set
# throws an error if it's not there

my_set.discard(value)
# removes value if it is in the set

my_set.pop()
# removes and returns an arbitrary element
\end{lstlisting}

\section{Graphs}
\subsection{Data Structure}
\begin{lstlisting}
class Edge():
	def __init__(self, edgefrom, edgeto, weight=1):
		self._from = edgefrom
		self._to = edgeto
		self._weight = weight

	def getFrom(self):
		return self._from

	def getTo(self):
		return self._to

	def getWeight(self):
		return self._weight

	def __lt__(self, other):
		return self._weight < other.getWeight()

	def __repr__(self):
		return "<Edge: "+str(self._from)+" -> "+str(self._to)+", weight="+str(self._weight)+">"

class Digraph():
	def __init__(self, vertecies):
		self._nodes = [set() for i in range(vertecies)]

	def add_vertex(self):
		self._nodes.append(set())

	def add_edge(self, edgefrom, edgeto):
		self._nodes[edgefrom].add(Edge(edgefrom, edgeto, 1))

	def adjacent(self, vertex):
		return iter(self._nodes[vertex])

	def degree(self, vertex):
		return len(self._nodes[vertex])

	def num_vertecies(self):
		return len(self._nodes)

	def num_edges(self):
		return sum(len(edges) for edges in self._nodes)

	def __str__(self):
		return str(self._nodes)

class UndirectedGraph(Digraph):
	def add_edge(self, edgefrom, edgeto):
		self._nodes[edgefrom].add(Edge(edgefrom, edgeto, 1))
		self._nodes[edgeto].add(Edge(edgeto, edgefrom, 1))

class WeightedDigraph(Digraph):
	def add_edge(self, edgefrom, edgeto, weight):
		self._nodes[edgefrom].add(Edge(edgefrom, edgeto, weight))

class WeightedUndirectedGraph(WeightedDigraph):
	def add_edge(self, v1, v2, weight):
		self._nodes[v1].add(Edge(v1, v2, weight))
		self._nodes[v2].add(Edge(v2, v1, weight))
\end{lstlisting}

\subsection{Algorithms}
\subsubsection{Breadth First Search}
\begin{lstlisting}
class BFS():
	def __init__(self, graph, start):
		self._distTo = [-1 for _ in range(graph.num_vertecies())]
		self._touched = [False for _ in range(graph.num_vertecies())]
		self._from = [-1 for _ in range(graph.num_vertecies())]
		self._start = start

		self._distTo[start] = 0
		self._touched[start] = True

		queue = deque()
		queue.append(start)

		while queue:
			vertex = queue.popleft()
			for v in graph.adjacent(vertex):
				if not self._touched[v.getTo()]:
					queue.append(v.getTo())
					self._distTo[v.getTo()] = self._distTo[vertex]+1
					self._touched[v.getTo()] = True
					self._from[v.getTo()] = vertex

	def isReachable(self, vertex):
		return self._touched[vertex]

	def distanceTo(self, vertex):
		return self._distTo[vertex]

	def pathTo(self, vertex):
		path = []
		pos = vertex
		while pos != self._start:
			path.append(pos)
			pos = self._from[pos]
		path.append(pos)
		path.reverse()
		return path
\end{lstlisting}

\subsubsection{Depth First Search}
\begin{lstlisting}
class DFS():
	def __init__(self, graph, start):
		self._distTo = [-1 for _ in range(graph.num_vertecies())]
		self._touched = [False for _ in range(graph.num_vertecies())]
		self._from = [-1 for _ in range(graph.num_vertecies())]
		self._start = start

		self._touched[start] = True
		self._distTo[start] = 0
		self.search(graph, start)

	def search(self, graph, node):
		for e in graph.adjacent(node):
			if not self._touched[e.getTo()]:
				self._touched[e.getTo()] = True
				self._from[e.getTo()] = node
				self._distTo[e.getTo()] = 1 + self._distTo[node]
				self.search(e.getTo())

	def isReachable(self, vertex):
		return self._touched[vertex]

	def distanceTo(self, vertex):
		return self._distTo[vertex]

	def pathTo(self, vertex):
		path = []
		pos = vertex
		while pos != self._start:
			path.append(pos)
			pos = self._from[pos]
		path.append(pos)
		path.reverse()
		return path
\end{lstlisting}

\subsubsection{Minumum Spanning Tree}
\begin{lstlisting}
class MST():
	def __init__(self, graph):
		self._touched = [False for _ in range(graph.num_vertecies())]
		self._v = graph.num_vertecies()
		self._edges = []

		pq = PriorityQueue()

		for e in graph.adjacent(0):
			pq.insert(e)
		self._touched[0] = True

		while pq.notEmpty():
			edge = pq.popmin()
			if not self._touched[edge.getTo()]:
				for e in graph.adjacent(edge.getTo()):
					if not self._touched[e.getTo()]:
						pq.insert(e)
				self._touched[edge.getTo()] = True
				self._edges.append(edge)

	def getGraph(self):
		graph = WeightedUndirectedGraph(self._v)
		for e in self._edges:
			graph.add_edge(e.getFrom(), e.getTo(), e.getWeight())
		return graph
\end{lstlisting}

\end{document}