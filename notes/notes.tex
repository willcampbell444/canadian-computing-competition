\documentclass[12pt]{article}

\title{Canadien Computing Contest Notes}
\author{Will Campbell}

\usepackage{amsmath}
\usepackage[margin=1in]{geometry}
\usepackage{indentfirst}

\usepackage{listings}
\usepackage{color}

\definecolor{dkgreen}{rgb}{0,0.6,0}
\definecolor{gray}{rgb}{0.5,0.5,0.5}
\definecolor{mauve}{rgb}{0.58,0,0.82}

\lstset{frame=tb,
  language=Python,
  aboveskip=3mm,
  belowskip=3mm,
  showstringspaces=false,
  columns=flexible,
  basicstyle={\small\ttfamily},
  numbers=none,
  numberstyle=\tiny\color{gray},
  keywordstyle=\color{blue},
  commentstyle=\color{dkgreen},
  stringstyle=\color{mauve},
  breaklines=true,
  breakatwhitespace=true,
  tabsize=3
}

\begin{document}

\section{Basic Containers}
\subsection{Lists}
\begin{lstlisting}
my_list = []

my_list[index]
# returns the item at the given index

my_list[-1]
# returns the last item,
my_list[-2]
# returns the second last item

my_list[start:end:step]
# returns a list of all items beginning at the given starting point up to but not including the end point, counting by intervals of step
# if start is left blank, it starts from the very beginning
# if end is left blank, it goes until the very end
# if step is left blank it goes by steps of +1

my_list.append(value)
# adds x to the end of the list

my_list.extend(other_list)
# adds all the items in other_list to the end of my_list

my_list.insert(index, value)
# adds an item into the list before the item at the given index

my_list.remove(x)
# removes the first item with value x
# if it does not exist an error will occour

my_list.pop(index)
# removes and returns the item at the given index
# if no index is given, the last item will be popped

my_list.clear()
# removes all items in the list

my_list.index(value[, start[, end]])
# returns the index of the first item with the given value
# error if none is found

my_list.count(value)
# returns the number of times the given value appears in the list

my_list.sort(key=None, reverse=False)
# sorts the list

my_list.sort(key=lambda x: x[2])
# sorts the list by the third element of each item in the list

my_list.reverse()
# reverses the list

my_list.copy()
# or
my_list[:]
# returns a shallow copy
\end{lstlisting}

\subsection{Stacks}
\begin{lstlisting}
stack = []

# push:
stack.append(value)

# pop:
stack.pop()
\end{lstlisting}

\subsection{Queues}
\begin{lstlisting}
from collections import deque

queue = deque()

# enqueue:
queue.append(value)

# dequeue:
queue.popleft()
\end{lstlisting}

\subsection{Priority Queues}
\subsubsection{Without Removal}
\begin{lstlisting}
import heapq

class PriorityQueueSimple(list):
	def insert(self, item, priority):
		heapq.heappush(self, (priority, item))

	def popmin(self):
		return heapq.heappop(self) # tuple of (priority, item)
\end{lstlisting}

\subsubsection{With Removal}
\begin{lstlisting}
import heapq

class PriorityQueue():
	def __init__(self):
		self._heap = []
		self._item_to_entry = {}
		self._counter = itertools.count()

	def insert(self, item, priority):
		if item in self._item_to_entry:
			self.remove(item)
		count = next(self._counter)
		entry = [priority, count, item]
		self._item_to_entry[item] = entry
		heapq.heappush(self._heap, entry)

	def popmin(self):
		while self._heap:
			priority, _, task = heapq.heappop(self._heap)
			if task is not None:
				del self._item_to_entry[task]
				return task
		raise KeyError("Pop from empty queue")

	def remove(self, item):
		entry = self._item_to_entry.pop(item)
		entry[-1] = None

	def __bool__(self):
		return bool(self._heap)

	def __repr__(self):
		return repr(self._heap)
\end{lstlisting}

\subsection{Dictionaries}

\subsection{Sets}

\section{Graphs}
\subsection{Data Structure}
\begin{lstlisting}
class Edge():
	def __init__(self, edgefrom, edgeto, weight=1):
		self._from = edgefrom
		self._to = edgeto
		self._weight = weight

	def getFrom(self):
		return self._from

	def getTo(self):
		return self._to

	def getWeight(self):
		return self._weight

	def __lt__(self, other):
		return self._weight < other.getWeight()

	def __repr__(self):
		return "<Edge: "+str(self._from)+" -> "+str(self._to)+", weight="+str(self._weight)+">"

class Digraph():
	def __init__(self, vertecies):
		self._nodes = [set() for i in range(vertecies)]

	def add_vertex(self):
		self._nodes.append(set())

	def add_edge(self, edgefrom, edgeto):
		self._nodes[edgefrom].add(Edge(edgefrom, edgeto, 1))

	def adjacent(self, vertex):
		return iter(self._nodes[vertex])

	def degree(self, vertex):
		return len(self._nodes[vertex])

	def num_vertecies(self):
		return len(self._nodes)

	def num_edges(self):
		return sum(len(edges) for edges in self._nodes)

	def __str__(self):
		return str(self._nodes)

class UndirectedGraph(Digraph):
	def add_edge(self, edgefrom, edgeto):
		self._nodes[edgefrom].add(Edge(edgefrom, edgeto, 1))
		self._nodes[edgeto].add(Edge(edgeto, edgefrom, 1))

class WeightedDigraph(Digraph):
	def add_edge(self, edgefrom, edgeto, weight):
		self._nodes[edgefrom].add(Edge(edgefrom, edgeto, weight))

class WeightedUndirectedGraph(WeightedDigraph):
	def add_edge(self, v1, v2, weight):
		self._nodes[v1].add(Edge(v1, v2, weight))
		self._nodes[v2].add(Edge(v2, v1, weight))
\end{lstlisting}

\subsection{Algorithms}


\end{document}